\documentclass{article}
\usepackage[utf8]{inputenc}
\usepackage[T1]{fontenc}
\usepackage{lmodern}
\usepackage[ngerman]{babel}
\usepackage{amsmath,amssymb,amsfonts,amsthm,mathtools,sansmath}
\usepackage{tikz-cd}
\usepackage{cleveref}
\usetikzlibrary{babel}

%Umgebungen
\newtheorem{thm}{Theorem}[section] 
\theoremstyle{definition}
\newtheorem{defn}[thm]{Definition}
\theoremstyle{plain}
\newtheorem{cor}[thm]{Korollar}
\newtheorem{lem}[thm]{Lemma}
\newtheorem{propo}[thm]{Proposition}
\newtheorem{axiom}[thm]{Axiom}
\theoremstyle{remark}
\newtheorem{remark}[thm]{Remark}
\newtheorem{example}[thm]{Example}
%commands 
\newcommand{\coker}{\mathsf{coker}}
\newcommand{\im}{\mathsf{im}}
\newcommand{\claim}{\textsf{Behauptung}:\hspace{0,2cm}}

%Aufgaben-Command
\newcommand{\aufgabe}[1]{
	{
		\vspace*{0.5cm}
		\textsf{\textbf{Aufgabe #1}}
		\vspace*{0.2cm}

	}
}
%Teilaufgabe
\newcommand{\teilaufgabe}[1]{
	{
		\vspace*{0.2cm}
		\textsf{\textbf{#1)}}
	}
}

\title{Aufgabenblatt 5}
\author{Philipp Stassen, Claas Latta}
\begin{document}
\maketitle
\aufgabe1
Sei $A$ ein Hauptidealring, $M$ ein endlich erzeugter $A$-Modul und
\begin{align}
	A^{r'}\oplus\bigoplus_{i=1}^{m}\bigoplus_{j=1}^{k_i}A/(q_i^{t_{ij}})\cong M \cong A^{r}\oplus\bigoplus_{i=1}^{n}\bigoplus_{j=1}^{l_i}A/(p_i^{s_{ij}})
\end{align}
mit $p_1,...,p_m,q_1,..,q_m\in A$ prim und $1\leq s_{i1}\leq ...\leq s_{il_i}$ resp. $1\leq t_{i1}\leq ...\leq t_{ik_i}$.

\teilaufgabe{i} Wir wollen zeigen, dass $r=r'$ 
\begin{proof}
Es ist $M_{tor}\cong\bigoplus_{i=1}^{m}\bigoplus_{j=1}^{k_i}A/(q_i^{t_{ij}})$, denn 
\begin{align}
	\prod_{i=1}^mq_i^{t_{ik_i}}\in\mathrm{Ann}_A\left(\bigoplus_{i=1}^{m}\bigoplus_{j=1}^{k_i}A/(q_i^{t_{ij}})\right)
\end{align}
 und $A^r$ ist frei, da $A$ ein Integritätsbereich ist.
 Also ist $A^{r'}\cong M/M_{tor}\cong A^r$. Nun folgt von Übungsblatt 3 Aufgabe 4, dass $r=r'$.
 \end{proof}

 \teilaufgabe{ii} \claim Bis auf Permutation der $q_i$ gilt, dass $p_i$ zu $q_i$ assoziiert ist. Insbesondere gilt dann $m=n$. 
 \begin{proof}
	 Es ist $M(q_i)\cong \bigoplus_{j=1}^{k_i}A/(q_i^{t_{ij}})$. Zu jedem $i$ haben wir den Annulator $a=q_i^{t_{ik_i}}\in \mathrm{Ann}_A(M(q_i))$. Für $m\in M(q_i)$ und $\varphi\in \mathrm{Aut}(M)$ muss also gelten, dass $0=\varphi(a\cdot m)=a\cdot \varphi(m)$. Also ist $a\in \mathrm{Ann}(\varphi(M(q_i)))$. 

	 Da $q_i$ prim ist, folgt aus $q_{i}^{t_{ik_i}}=a=p_{i}^{t_{il_i}}$ bereits, dass $q_i$ assoziiert zu $p_i$ ist. 
 \end{proof}
 \teilaufgabe{iii} \claim Ist $p_i$ assoziiert zu $q_i$, dann ist $s_{ij}=t_{ij}$ für alle $j$.
\begin{proof}
	Vermöge Teilaufgabe ii) wissen wir, dass $\varphi(M(q_i))=M(p_i)$. Es genügt also die Aussage für einen primären Modul zu beweisen.
	
	Sei $\bigoplus_{j=1}^kA/(q^{t_{j}})\cong M\cong\bigoplus_{j=1}^lA/(p^{s_{j}})$ mit dem Isomorphismus 
	\begin{align}
		\varphi:\bigoplus_{j=1}^kA/(q^{t_{j}})\to \bigoplus_{j=1}^lA/(p^{s_{j}}).
	\end{align} 
	Wir wissen, dass 
	\begin{align}
		(q^{t_k})= \mathrm{Ann}(M)= (p^{s_l}).
	\end{align}
	Deshalb folgt, dass $t_k = s_l$. Desweiteren induziert $\varphi$ einen Isomorphismus 
	\begin{align}
		M/(A/q^{t_k})\cong \bigoplus_{j=1}^{k-1}A/(q^{t_{j}})\cong \bigoplus_{j=1}^{l-1}A/(p^{s_{j}})\cong N/(A/p^{s_l}).
	\end{align}
	Wir iterieren dieses Verfahren und erhalten die Eindeutigkeit der Exponenten.
	\end{proof}
\teilaufgabe{iv} \claim Es existieren minimales $d\in\mathbb{N}$ und bis auf Assoziiertheit eindeutige $a_1,...,a_d\in A\backslash\{0\}$ 
\begin{align}
	M\cong A^r\oplus\bigoplus_{i=1}^dA/(a_i)
\end{align}
und $a_1|a_2|...|a_d$.
\begin{proof}
	Nach dem chinesischen Restsatz (siehe Wiki oder irgendein Algebrabuch) ist für teilerfremde $a_1,...,a_n$
	\begin{align}
		A/(a_1\cdot ...\cdot a_n)\cong \prod_{i=1}^nA/(a_i)\cong \bigoplus_{i=1}^nA/(a_i)
	\end{align}
	Wir definieren 
	\begin{align}
		a_d&:=\prod_{i=1}^n p_i^{t_{il_i}} \\
		a_h&:= \prod_{i=1}^n p_i^{t_{i(l_i-(d-h)}} \text{ für } h<d
	\end{align}
	wobei wir definieren, dass $t_{ij}=1$ falls $j\leq 0$.
	Dann gilt nach dem chinesischen Restsatz
	\begin{align}
		A^{r}\oplus\bigoplus_{i=1}^{n}\bigoplus_{j=1}^{l_i}A/(p_i^{s_{ij}})\cong A^{r}\oplus\bigoplus_{i=1}^dA/(a_i),
	\end{align}
	wobei $d=\mathrm{max}_{i=1}^n(l_i)$.

	Dass diese Zerlegung eindeutig ist bis auf Assoziiertheit folgt daraus, dass die $p_i$ eindeutig sind bis auf Assoziiertheit. Es ist nach Konstruktion klar, dass $a_1|a_2|...|a_d|$.
\end{proof}
\end{document}
